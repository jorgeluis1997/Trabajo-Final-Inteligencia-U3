\section{RESUMEN} 

El incremento en el número de libros, y otros hace que los sistemas tradicionales de búsqueda de literatura sobre algún tema en particular sean complejos y lentos, no siempre obteniendo buenos resultados. Peor aún si se trata de recomendar algún libro en particular, basado en el conocimiento de la calidad de su contenido. 
\\
Por ello se propone diseñar e implementar un sistema de recomendación de libros. Se diseñó un modelo basado en lenguaje de programación Python para la recomendación de libros con el objetivo de la recomendación de varios tipos de libros del interés del usuario, mientras la recomendación de libros busca incrementar el conocimiento de los usuarios, este sistema ayuda a seleccionar un tema de investigación o encontrar una referencia bibliográfica ajustada al tema de investigación del usuario. 


\section{¿Qué es Colap?} 
Convertir datos en información es, hoy en día, una ventaja competitiva que las empresas deben comenzar a explotar. Optimizar sus procesos, entender su entorno o adelantarse a futuras tendencias son solo algunas de las posibilidades que brindan las herramientas de análisis de información.
Colap es una herramienta de análisis y visualización de información con la particularidad que está diseñada para personas que no son del área IT. Se consulta escribiendo en español (similar a como se hace una búsqueda en Google), permite crear reportes personalizados y además incluye la posibilidad de compartir consultas con otros miembros de la empresa.

