\section{MARCO TEORICO} 


\subsection{Python}
	
Python es un lenguaje sencillo y rápido de aprender. Su sintaxis es parecida a escribir cualquier texto en inglés, pero con la potencia de sus principales competidores en el BackEnd.
Es un placer de leer y redactar. Python predica que un código debe ser escrito por humanos para humanos. Después de todo lo que programas va a ser leído por ti y por el resto del equipo. Si escribes para máquinas, solo te entenderán máquinas.
\\
Además, viene con “Pilas incluidas”. Eso quiere decir que posee su propio gestor de paquetes, sin necesidad de instalar aplicaciones externas. Simplificando tareas de instalación o actualización.
\\
Otro punto a su favor es que no necesita un ecosistema para ejecutarse, como puede ser Xampp, Vangrant, Docker… Python solo requieres Python. Lanzando un comando en el terminal estará ejecutándose su propio servidor Web, consiguiendo que su puesta en producción sea sorprendentemente rápida.
Y por si fuera poco, es el segundo lenguajes que mejor esta pagado por las empresas. Por detrás de Ruby.

\subsection{Google Colab: Python y Machine Learning en la nube}
En este veremos qué es y cómo utilizar Google Colab, la herramienta de Google en la nube para ejecutar código Python y crear modelos de Machine Learning a través de la nube de Google y con la posibilidad de hacer uso de sus GPU . Sí, has leído bien: con sus GPU y en la nube.


		
\subsection{Frameworks Web}

Entre sus numerosos y fantásticos Frameworks, nos podemos encontrar unas bestias: Django y Flask (que no confundir que el zombie Adobe Flash).
Django sería lo más cercano a Laravel en PHP o Ruby on Rails para Ruby. Un marco de trabajo completo y eficiente para desarrollar Aplicaciones Web de una gran complejidad con un mínimo esfuerzo. Casi cualquier cosa que necesites posiblemente estará integrada.
\\
Para desarrollos altamente personalizados o con unos tiempos cortos, nos encontramos a Flask. Autodenominado microframework, pero con funcionalidades sencillas e inteligentes para construir cualquier sitio que se te pase por la cabeza.
Uno no sustituye al otro. Merece la pena experimentarlos y ver sus diferentes enfoques. 


	
\subsection{Ventajas de programar en Python}
\begin{itemize}	
\item	Simplificado y rápido: Este lenguaje simplifica mucho la programación, es un gran lenguaje para scripting.\\
\item	Elegante y flexible: El lenguaje ofrece muchas facilidades al programador al ser fácilmente legible e interpretable.\\
\item	Programación sana y productiva: Es sencillo de aprender, con una curva de aprendizaje moderada. Es muy fácil comenzar a programar y fomenta la productividad.\\
\item	Ordenado y limpio: es muy legible y sus módulos están bien organizados.\\
\item	Portable: Es un lenguaje muy portable. Podemos usarlo en prácticamente cualquier sistema de la actualidad.\\
\item	Comunidad: Cuenta con un gran número de usuarios. Su comunidad participa activamente en el desarrollo del lenguaje.\\


	

\subsection{Futuro}

Las previsiones son muy buenas. Las versiones son constantes y compatibles con todas las plataforma. Su creador, Guido van Rossum, es denominado como “Benevolente dictador vitalicio” por dejar que la comunidad tomen las decisiones. Tan solo dejó 4 directrices:
\item	Python debería ser fácil, intuitivo y tan potente como sus principales competidores.
\item	El proyecto sería de Código Abierto para que cualquiera pudiera colaborar.
\item	El código escrito en Python sería tan comprensible como cualquier texto en inglés.
\item	Python debería ser apto para las actividades diarias permitiendo la construcción de prototipos en poco tiempo.

\end{itemize}

	